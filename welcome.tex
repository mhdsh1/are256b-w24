\documentclass{beamer}
\usetheme{default}

% setting allowframebreaks as default
\let\oldframe\frame
\renewcommand\frame[1][allowframebreaks]{\oldframe[#1]}

\title{Welcome to ARE 256B Sections!} 
\author{Mahdi Shams}
\date{\today}
%\logo{\large \LaTeX{}}

\begin{document}

%----------------------------------------------------------------------------%
\begin{frame}
%----------------------------------------------------------------------------%
\titlepage 
\end{frame}

% Remove logo from the next slides
%\logo{}

%----------------------------------------------------------------------------%
\begin{frame}{Outline}
%----------------------------------------------------------------------------%
\tableofcontents
\end{frame}

%----------------------------------------------------------------------------%
\begin{frame}{Introduction}
%----------------------------------------------------------------------------%
\begin{itemize}
\item I'm \href{https://www.youtube.com/watch?v=J9wRl5TVFvc}{Mahdi} Shams,
 your TA for this course. 
\item I'm a second-year PhD student in Davis ARE. 
\item Originally from Tehran, Iran, I pursued my undergraduate studies in 
engineering there and later earned my master's degree in economics in 
Toulouse, France.
\item I'm interested in the intersection of environmental policy and public 
economics, and I believe econometrics plays a crucial role in my work. 
  
%\item English is not my first language. My mother tongue is 
%[Farsi (Persian)](https://simple.wikipedia.org/wiki/Persian_language). 
% Clarity is important to me, so please let me know if you notice any 
% unusual pronunciations. 
  
  \item I'm here to assist you, so feel free to reach out with any questions or 
concerns!
 \item my email is mashams[at]ucdavis[dot]edu.
  
\end{itemize}

\end{frame}
%----------------------------------------------------------------------------%
\begin{frame}{Introduction}
%----------------------------------------------------------------------------%
\begin{center}
\Huge It's your turn now!
\end{center}

\end{frame} 
%----------------------------------------------------------------------------%
\begin{frame}{Announcements}
%----------------------------------------------------------------------------%
Announcements:
\begin{itemize}
\item Sections: Fridays 9:00-9:50 am at Veihmeyer Hall 116
\item Mahdi OHs: Fridays 10:00-11:00 am at SSH 2136  
\end{itemize}
\end{frame}

%----------------------------------------------------------------------------%
\begin{frame}{Setup}
%----------------------------------------------------------------------------%

Access to Stata:
\begin{itemize}
\item option 1: https://stata-support.ucdavis.edu/
\item option 2: https://virtuallab.ucdavis.edu/
\item option 3: ARE Computer Lab
\end{itemize}
\end{frame}

%----------------------------------------------------------------------------%
\begin{frame}{Week 1: Stata Basics}
%----------------------------------------------------------------------------%

\begin{itemize}
\item type \texttt{doed} in the command window to open the do-file editor
\item asking help 1: \texttt{help} "command"
\item asking help 2: google \emph{help "command" stata}
\item basic stata syntax: \texttt{command varlist if in, options}
\end{itemize}

\begin{itemize}
\item setting working directory
\item importing data
\item \texttt{browse, describe}, \dots
\item operators
\item getting summary statistics: \texttt{summ, tabulate|, \dots}
\item \texttt{gen, replace, drop, keep, \dots}
\item using functions: \texttt{log(x)}
\end{itemize}

\end{frame}


%----------------------------------------------------------------------------%
\begin{frame}{Week2: Lectures 1 to 3 (limited dep. variable) and presentation}
%----------------------------------------------------------------------------%

Lectures
\begin{enumerate}[1.]
\item estimating linear models
\item estimating probit models
\item plotting the scatter plot
\item computing the rmse
\end{enumerate}

Presentation
\\
Motivation: Fixed costs + now you have more time
\\
\begin{enumerate}[1.]
\item template do file
\item making log file and converting it to pdf
\item making the tex file (look at the example.tex)
\end{enumerate}

\end{frame}
%----------------------------------------------------------------------------%
\begin{frame}{Week3}
%----------------------------------------------------------------------------%
\begin{itemize}
\item creating a random subsample 
\item Censoring 
% insert the graph 
\item Sample Selection
% insert the graph (maybe from the lecture)
\item exporting plots
\item exporting regression tables using estout
\item HA1 Questions!
\end{itemize}
\end{frame}

\begin{frame}{Censoring}
\begin{itemize}
\item $Y$ is known exactly if some criterion defined in terms of $Y$ is met.
\item X variables are observed for the entire sample
\item Example: Determinants of income; income is measured exactly only if it 
is above the poverty line. All other incomes are reported at the poverty line 
(the lower threshold).
\end{itemize}
\end{frame}


\begin{frame}{Censoring}
\begin{itemize}
\item is observed only if a criteria defined in terms of some other random
 variables (B) is met (e.g. In our example, the criteria is employment status).
\item We observe the determinants of B (which we call by Q) for the entire 
sample.
\item Example: Survey data with item or unit non-response
\end{itemize}
\end{frame}


%----------------------------------------------------------------------------%
\begin{frame}{Links}
%----------------------------------------------------------------------------%

\begin{itemize}
\item \href{https://github.com/mhdsh1/are256b-w24/blob/main/example.do}
{Example .do file} 
\item \href{https://github.com/mhdsh1/are256b-w24/blob/main/example.tex}
{Example .tex file}
\item \href{https://www.stata.com/support/faqs/graphics/gph/stata-graphs/}
{Stata Visual overview for creating graphs}
\item \href{https://repec.sowi.unibe.ch/stata/estout/index.html}
{exporting regression tables using estout}
\item \href{https://www.overleaf.com/learn/latex/Learn_LaTeX_in_30_minutes}
{\LaTeX in 30 Minutes}
\end{itemize}

\end{frame}
\end{document}